\chapter{一些使用技巧}

如果在生成文档时发生错误,不要惊慌,可以先把生成的文件全部删除再试一次。
就是把除了tex文件外的其它同名文件都删掉。

使用WinEdt编辑tex文件时,如果嫌命令太长打着费劲,试试只输前几个字母然后按
“Ctrl+Enter”键,哈!WinEdt替你把剩下的部分补全了。

遇到问题不要慌,看下方小窗口里提示的出错信息,会有很多提示你错在哪里的。

不同系统下生成的eps可能会有兼容问题,
如本模版中的setroot.eps和rffndb.eps,在xp和windows 7 x64似乎不能通用。
解决方案很简单,
只要用bmeps -p 1 -c setroot.jpg setrooteps重新生成一次即可解决。
rffndb.eps生成命令同setroot.eps。

这个文档我用的是gVim编辑的,gVim自带的自动补全功能比WinEdt更强大,让我在
编写这个文档时省了不少重复工作量。

如果会使用make程序,那么使用Makefile来生成文档更方便一些。

在UTF-8版本中,如果一个命令后紧跟汉字,比如像这样“\verb+songti好的+”,
编译的时候就会报错,处理办法就是在命令后面加一个空格或者一个大括号,就像这样:
“\verb+songti 好的+”或者“\verb+songti{}好的+”

差不多了,就写这几条吧,想起来什么再写。

把另外几个参考文献当引用例子使用一下:专利\cite{WangZL},标准\cite{WangStd},
电子文档\cite{ZLB:1997},期刊文章\cite{LUOZ:2007},
学位论文\cite{wang:2008,wangmt:2008}。

这份文档从规划到完成,历时近20日,也是自己\LaTeX 学习一个总结吧。
